\chapter{Marco teórico}

\noindent El objetivo de este análisis es introducir al cómputo en paralelo y distribuido y describir las capacidades principales de los \textit{frameworks}  de \textit{Big Data}: \textit{Spark} y \textit{Dask}.

\newpage

\section{Introducción al cómputo en paralelo}

\section{Apache Spark}

Apache Spark es un motor de cómputo unificado con bibliotecas para procesamiento de datos en paralelo en clústeres de computadoras. Spark soporta múltiples lenguajes de programación como Python, Java, Scala y R. Entre sus aplicaciones más compunes están trabajos de SQL, \textit{streaming} y \textit{machine learning}.\cite{sparkguide}

\section{Arquitectura de Spark}

Una aplicación de Spark consiste de un proceso \textit{driver} y un conjunto de procesos llamados \textit{executor}. Ambos tipos de procesos trabajan en conjunto durante la ejecución de la aplicación. El proceso \textit{driver} es la parte central de la aplicación, ya que se encarga de mantener información sobre la aplicación, responder a las instrucciones del usuario y analizar, distribuir y calendarizar el trabajo entre los ejecutores. Este proceso existe en uno solo de los nodos y está en constante comunicación con los demás. \cite{sparkguide}. Los \textit{executors}, por otra parte, tienen el objetivo de realizar el trabajo que el \textit{driver} les asigna. Estos se encargan de ejecutan el trabajo y guardan datos para la aplicación \cite{sparkclusteroverview}.

La abstracción de datos principal en Spark son los \textit{Resilient Distributed Datasets (RDDs)} que son colecciones de objetos particionadas en un clúster y que se pueden manipular en paralelo a través de transformaciones. Además, estas transformaciones tienen un \textit{lazy evaluation}, es decir que no realizan la acción inmediatamente, sino que la añaden a un plan eficiente de ejecución. Al llamar una acción, que es una instrucción para calcular un resultado a partir de las transformaciones del plan \cite{sparkguide}.  

\section{Dask}

Dask es una librería de Python para cómputo en paralelo que extiende herramientas populares en el análisis de datos como \textit{NumPy}, \textit{Pandas} y \textit{Python iterators} a tareas que no caben en memoria o a ambientes distribuidos. Además, cuenta con un calendarizador dinámico de tareas y está desarrollado totalmente en \textit{Python}. Se puede utilizar en modo \textit{standalone} o en clústeres con cientos de máquinas. Dask ofrece una interfaz familiar ya que emula a \textit{Numpy} y \textit{Pandas}. 

\subsection{Cambios cerebrales y genéticos}

\noindent Brizendine (2010) escribe que algunos científicos piensan que ciertas áreas del cerebro son como centros de actividad que mandan señales eléctricas a otras áreas del cerebro ocasionando un determinado comportamiento.\footnote{ Por ejemplo, en el hombre la corteza del cíngulo anterior pesa opciones, detecta conflicto y motiva decisiones. La unión temporoparietal busca soluciones rápidas y ante situaciones estresantes toma en cuenta la perspectiva de otros individuos. La corteza cingulada anterior rostral se encarga de procesar los errores sociales, como la aprobación o desaprobación de otros.}

\begin{quote}
    \small{Mientras que la distinción entre los cerebros de niños y niñas empieza biológicamente, estudios recientes muestran que es \textit{solo} el comienzo. La estructura cerebral no está escrita sobre piedra en el nacimiento ni al final de la infancia, como antes se creía, sino que continúa cambiando a lo largo de la vida. Más que ser inmutable, nuestros cerebros son mucho más plásticos y cambiables de lo que los científicos creían hace una década. El cerebro humano es también la máquina de aprendizaje más talentosa que conocemos. Así que nuestra cultura y el cómo nos enseñaron a comportarnos desempeñan un papel importante en el diseño y reestructura de nuestros cerebros (Brizendine 2010, 5-6).}
\end{quote}

% Para citas muy largas es mejor el \begin{quote}

\vspace{1em}
\noindent \textbf{Hipótesis 3.} \hfill\begin{minipage}{\dimexpr\textwidth-3cm}
\textit{La intensidad religiosa está relacionada negativamente con la innovación.}
\end{minipage}
\vspace{1em}

% Para plantear hipótesis

\begin{table}[H]
\centering
\caption{Índices de modernidad y tradicionalismo}
\label{PHEL}
\begin{tabular}{|ccc|}
\hline
 País & Índice de modernidad & Índice de tradicionalismo  \\ 
\hline
Alemania & 0.58 & 0.45 \\
Austria & 0.55 & 0.49  \\
Bélgica & 0.50 & 0.49  \\
Canadá & 0.61 & 0.50 \\
Dinamarca & 0.58 & 0.44 \\ 
España & 0.47 & 0.62 \\
Estados Unidos & 0.59 & 0.44  \\
Finlandia & 0.62 & 0.38 \\
Francia & 0.49 & 0.59 \\ 
Holanda & 0.58 & 0.49 \\ 
Irlanda & 0.54 & 0.59  \\
Islandia & 0.63 & 0.54 \\
Italia & 0.56 & 0.58  \\
Japón & 0.42 & 0.48 \\
Noruega & 0.53 & 0.44 \\
Portugal & 0.50 & 0.71 \\ 
Reino Unido & 0.56 & 0.54  \\
Suecia & 0.62 & 0.51 \\
Promedio & 0.58 & 0.51 \\
\hline
\end{tabular}

\begin{tabular}{c}
\footnotesize{Fuente: Bojilov y Phelps (2012).}
\end{tabular}

\end{table}

% Para diseñar tablas

\section{Dask}


