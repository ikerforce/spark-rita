\chapter{Metodología}

\noindent Este capítulo describe el proceso que se siguió para comparar el desempeño de \textit{Dask} y \textit{Spark} bajo condiciones homologadas, en dos ambientes, uno local y otro distribuido en la nube. Para tener una mayor visibilidad del desempeño de cada una de ellos, en distintos escenarios, se crearon 8 procesos distintos que consisten en filtración, agrupación, conteos y cálculo de estadísticas sobre los datos seleccionados. Los 8 procesos fueron escritos de forma independiente en \textit{Spark} y \textit{Dask} utilizando funciones equivalentes en ambos \textit{frameworks}. También se incorporó una variación en el tamaño de las muestras para observar el desempeño de cada herramienta con distintos volúmenes de datos. En total se realizaron 14,280 ejecuciones buscando ejecutar 100 veces cada combinación de muestra y ambiente (existieron algunas excepciones que se especifican más adelante debido a falta de recursos o ejecuciones demasiado prolongadas). Cada ejecución fue registrada en una base de datos para su análisis posterior.
\newpage

\section{Procesos para comparación de desempeño}

Para contrastar el desempeño de las herramientas, se implementaron 8 procesos que realizan distintas operaciones sobre el conjunto de datos. Estos procesos buscan someter a los \textit{frameworks} a diferentes escenarios para los que se usan comúnmente estas herramientas. Entre estas operaciones se estuvieron las siguientes:
\begin{itemize}
	\item Cálculo de agregaciones sencillas como el conteo de elementos u obtención de los valores máximo y mínimo.
	\item Cálculo de estadísticas como el promedio y la desviación estándar.
	\item Escritura a bases de datos \textit{SQL} o archivos \texttt{parquet}.
	\item Limpieza de datos como eliminación de nulos y borrado de duplicados.
\end{itemize}

Además se implementó un proceso de cálculo de ruta mínima que involucró muchas de las operaciones listadas. En las siguientes secciones se explica de forma más detallada cada uno de los procesos y se listan las operaciones que los componen.

Para ejecutar los datos se creó un \textit{script} que sigue el proceso descrito en el algoritmo \ref{ejecucion_procesos} para la ejecución de los distintos procesos. Este \textit{script} tuvo como entradas la muestra de datos (\texttt{muestra}), el número de ejecuciones a realizar (\texttt{n}) y la lista de procesos que se ejecutaron (\texttt{procesos}). Además, dentro del algoritmo, se definió la lista \texttt{framework} compuesta de unos y ceros con un orden aleatorio que indicaba si el proceso iba a ser ejecutado con \textit{Dask} (en caso de cero) o con \textit{Spark} (en caso de uno). 

\begin{figure}
\begin{algoritmo}[H]
\caption{Ejecución de procesos}\label{ejecucion_procesos}
\begin{algorithmic}[1]
\Procedure{Ejecuciones(muestra, n, procesos)}{}
\For{\texttt{\textit{proceso} in \textit{procesos}}}
	\State $\textit{framework} \gets \text{lista de 0 y 1 en orden aleatorio de longitud 2\textit{n}}$
	\For{\texttt{\textit{x} in \textit{framework}}}
		\If {\texttt{\textit{x} = 1}}
		\State \texttt{Ejecutar el \textit{proceso} en \textit{Spark}}
		\Else
		\State \texttt{Ejecutar el \textit{proceso} en \textit{Dask}}
		\EndIf
	\EndFor
\EndFor
\EndProcedure
\end{algorithmic}
\end{algoritmo}
\caption{Proceso de ejecución del experimento.}
\end{figure}

De esta forma, cada proceso se ejecutó $n$ veces en cada \textit{framework} y las ejecuciones entre \textit{Spark} y \textit{Dask} se alternaron. Además, para iniciar la ejecución de un proceso fue necesario que el anterior finalizara, lo que evitó que compitieran por recursos. El algoritmo \ref{ejecucion_procesos} se ejecutó 9 veces, 4 de forma local con tamaños de muestra distintos y 5 en un ambiente distribuido en la nube con los mismos tamaños de muestra que la prueba local y una adicional conformada del total de los datos que no fue ejecutada localmente debido su inestabilidad, que provocó tiempos de ejecución excesivos o errores que evitaron la finalización de los procesos. En la mayoría de los casos el valor de $n$ fue $100$. Sin embargo, en algunas muestras este fue reducido para procesos específicos debido a que la ejecución era muy tardada o presentaba errores.

Adicionalmente, para verificar que los resultados de los procesos de \textit{Dask} y \textit{Spark} tuvieran el mismo resultado, se utilizó un \textit{script} que leía el resultado de cada proceso ejecutado en un \textit{framework} y lo comparaba con su contraparte. De esta forma, se aseguró que el resultado fuera el mismo. No obstante, es importante mencionar que para los procesos que utilizan la desviación estándar, sobre una muestra de datos, puede haber diferencias ya que \textit{Dask} y \textit{Spark} generaron muestras distintas. El proceso de revisión de resultados se puede consultar en \cite{compara-resultados}.

\subsection{Proceso 1: Cálculo del tamaño de la flota por aerolínea}

Este proceso calculó el número de aviones activos con los que contaba una aerolínea en un tiempo específico. La agregación se hizo por día, mes, trimestre y año. Para obtener el número de equipos únicos se contó el número de matrículas distintas correspondientes a cada aerolínea. El proceso estuvo compuesto de las siguientes operaciones:

\begin{itemize}
	\item Lectura de datos desde \texttt{parquet}.
	\item Borrado de duplicados.
	\item Conteo.
	\item Agregación de datos agrupando por diferentes columnas.
	\item Escritura de datos a \textit{MySQL}.
\end{itemize}

\subsection{Proceso 2: Demoras por aerolínea}

Este proceso tuvo como objetivo obtener el promedio de tiempo de demora de cada aerolínea agrupando por las columnas correspondientes a los periodos: año, trimestre, mes y día. Este proceso realizó las siguientes operaciones principales:

\begin{itemize}
	\item Lectura de datos desde \texttt{parquet}.
	\item Conteo.
	\item Cálculo de promedio.
	\item Agregación de datos agrupando por diferentes columnas.
	\item Escritura de datos a \texttt{parquet}.
\end{itemize}

\subsection{Proceso 3: Demora máxima por aeropuerto de origen}

Este proceso calculó el tiempo máximo de demora agrupando por el aeropuerto de origen y obteniendo el resultado para los periodos por año, trimestre, mes y día. El proceso estuvo compuesto de las siguientes operaciones:

\begin{itemize}
	\item Lectura de datos desde \texttt{parquet}.
	\item Conteo.
	\item Cálculo del máximo.
	\item Agregación de datos agrupando por diferentes columnas.
	\item Escritura de datos a \textit{MySQL}.
\end{itemize}

\subsection{Proceso 4: Demora mínima por aeropuerto de destino}

Este proceso fue similar al anterior, con las diferencias de que calculó el tiempo mínimo de demora y que hizo la agrupación de acuerdo al aeropuerto de destino. Esto obteniendo el resultado para los periodos por año, trimestre, mes y día. El proceso estuvo conformado de las siguientes operaciones:

\begin{itemize}
	\item Lectura de datos desde \texttt{parquet}.
	\item Conteo.
	\item Cálculo del mínimo.
	\item Agregación de datos agrupando por diferentes columnas.
	\item Escritura de datos a \texttt{parquet}.
\end{itemize}

\subsection{Proceso 5: Demora promedio por ruta entre aeropuertos}

Este proceso obtuvo el tiempo promedio de demora para cada ruta entre los aeropuertos. El primer paso fue obtener la ruta que se recorrió en cada vuelo y después se calculó el promedio de las demoras en el periodo correspondiente.

\begin{itemize}
	\item Lectura de datos desde \texttt{parquet}.
	\item Concatenación de columnas.
	\item Conteo.
	\item Cálculo de promedio.
	\item Agregación de datos agrupando por diferentes columnas.
	\item Escritura de datos a \texttt{parquet}.
\end{itemize}

\subsection{Proceso 6: Desviación estándar de la demora por ruta entre \texttt{market ids}}

Este proceso obtuvo la desviación estándar del tiempo de demora para cada ruta entre áreas identificadas por el indicador \texttt{market id} que corresponde a las áreas geográficas cuyos aeropuertos sirven al mismo mercado de personas. El primer paso fue obtener la ruta que se recorrió en cada vuelo y después se calculó la desviación estándar de los indicadores de demoras en el periodo correspondiente.

\begin{itemize}
	\item Lectura de datos desde \texttt{parquet}.
	\item Concatenación de columnas.
	\item Conteo.
	\item Cálculo de la desviación estándar.
	\item Agregación de datos agrupando por diferentes columnas.
	\item Escritura de datos a \textit{MySQL}.
\end{itemize}

\subsection{Proceso 7: Eliminación de nulos}

Este proceso tuvo como objetivo eliminar todos los registros que tuvieran algún valor nulo en cualquiera de las columnas seleccionadas. 

\begin{itemize}
	\item Lectura de datos desde \texttt{parquet}.
	\item Eliminación de nulos.
	\item Conteo.
\end{itemize}

\subsection{Proceso 8: Cálculo de la menor ruta en tiempo utilizando el algoritmo de ruta mínima \textit{Dijkstra}}

El último proceso consistió en calcular la ruta mínima (minimizando el tiempo transcurrido entre la salida del origen y la llegada al destino final) entre dos aeropuertos cualesquiera. Para esto se hizo una implementación del algoritmo \ref{dijkstra_modificado} utilizando \textit{DataFrames} junto con las operaciones predefinidas en cada uno de los \textit{frameworks} (como \texttt{min}, \texttt{orderBy}, \texttt{filter} y operaciones aritméticas) para manipularlos y una función definida por el usuario (\texttt{udf}) que se utilizó para cambiar el formato de la fecha de los vuelos.

El proceso estuvo compuesto de las siguientes operaciones principales:

\begin{itemize}
	\item Lectura de datos desde \texttt{parquet}.
	\item Filtrado de datos.
	\item Conteo de datos.
	\item Borrado de duplicados.
	\item Cálculo de registro mínimo.
	\item Ordenamiento de datos.
	\item *Almacenamiento temporal de datos mediante \texttt{persist} (\textit{Dask}) y \texttt{checkpoint} (\textit{Spark}).
	\item Suma de columnas a un escalar.
	\item Concatenación de \textit{DataFrames}.
	\item Escritura de datos a \textit{MySQL}.
\end{itemize}

Las operaciones con asterisco variaron debido a que se utilizó, para cada \textit{framework}, la operación que mejor desempeño tenía en cuanto a tiempo para almacenar datos intermedios. En el caso de \textit{Spark}, se usó la operación \texttt{checkpoint}, ya que se deshacía de la parte del plan de ejecución que ya había sido ejecutada y almacenaba los datos intermedios, liberando memoria (a diferencia de \texttt{cache}). Para \textit{Dask}, por otro lado, resultó más conveniente usar la función \texttt{persist} ya que \textit{Dask} no ejecutaba otra vez el plan de ejecución completo sino a partir del último \texttt{persist}.

\subsection{Registro de tiempo y resultados}

Al finalizar la ejecución de cada proceso, se obtuvo el tiempo de ejecución del mismo (\textit{duration}). Después, esta información se escribió en una base de datos \textit{SQL}¸ junto con los detalles del ambiente en que se ejecutó (\textit{description} y \textit{resources}), la fecha de inicio (\textit{start\_ts}) y término de la ejecución (\textit{end\_ts}) en tiempo UNIX, la fecha de inserción del registro (\textit{insertion\_ts}), el tamaño de muestra en el que se ejecutó el proceso (\textit{sample\_size}), el ambiente en el que se ejecutó el proceso (local o en la nube) y el nombre del proceso que se ejecutó (\textit{process}). La siguiente tabla muestra un ejemplo de la información que se almacenó en la base de datos.

\begin{figure}
\begin{center}
\begin{tabular}{|c|c|}
 \hline
  process & demoras\_ruta\_mktid\_dask \\ 
  start\_ts & 1615231907.4139209 \\
  end\_ts & 1615231916.5688508 \\ 
  duration & 9.154929876327515 \\ 
  description & Ejecucion de prueba en equipo local. \\
  resources & 16 GB y 12 nucleos. \\
  sample\_size & 10K \\
  env & local \\
  insertion\_ts & 2021-03-08 13:31:56 \\
  \hline
\end{tabular}
\end{center}
\caption{Ejemplo de la información que se registró al terminar la ejecución de una prueba del experimento.}
\end{figure}


La información de la ejecución se registró en una de dos tablas dependiendo del \textit{framework} elegido. Los resultados se almacenaron en una tabla específica que se sobreescribió, en el caso de \textit{MySQL}, con la finalización de la ejecución de el proceso correspondiente o se almacenaron en uno o más archivos \texttt{parquet} (dependiendo del número de particiones), en el caso de los procesos que tenían este tipo de archivo como destino. 

\section{Infraestructura}

Los \textit{frameworks} fueron comparados en dos ambientes distintos para probar su funcionamiento en modo \textit{standalone} y distribuido en un grupo de computadoras. El primer ambiente fue el local y el segundo fue el distribuido en la nube. Las siguientes secciones explicarán las características de los ambientes y las ejecuciones en ellas.

\subsection{Ambiente local}

A pesar de que las computadoras personales se quedan cortas para realizar grandes cargas de trabajo de \textit{Big Data}, siguen siendo una herramienta importante para hacer análisis previo y realizar cargas de trabajo no muy grandes. Tanto \textit{Dask} como \textit{Spark} tienen la capacidad de funcionar en una sola máquina y aprovechar los recursos para realizar las cargas de trabajo de forma más rápida que otras herramientas. Por esta razón fue importante probar ambos \textit{frameworks} en un ambiente local. 

Para ello se utilizó un equipo con las siguientes características:

\begin{center}
\begin{tabular}{|c|c|}
 \hline
  Procesador & Intel Core i7-10750H CPU @ 2.60GHz \\ 
  Núcleos & 12 \\
  Memoria & 15.5 GiB \\ 
  Almacenamiento & 128 GB (SSD) + 1 TB (HDD) \\ 
  Sistema Operativo & Ubuntu 20.04.2 LTS \\
  \hline
\end{tabular}
\end{center}

Los procesos fueron ejecutados mediante el proceso descrito en \ref{ejecucion_procesos} y almacenados en una base de datos \textit{MySQL} versión \texttt{8.0.25-0 ubuntu0.20.04.1} alojada en el mismo equipo y compartiendo los recursos listados anteriormente.

Además, se utilizó un ambiente de \textit{Anaconda} con \textit{Python} \texttt{3.8.5}. Este ambiente se utilizó para las ejecuciones con la versión  \texttt{2021.1.1} de \textit{Dask} y la versión \texttt{2.4.7} de \textit{Spark}. 

\subsection{Ambiente en la nube}

El ambiente en la nube tuvo como objetivo probar el desempeño de \textit{Dask} y \textit{Spark} en un ambiente distribuido y homogéneo. Para ello se utilizó un grupo de 4 nodos (1 \textit{head} y 3 \textit{worker}) en \textit{Amazon Elastic Map Reduce (EMR)}. La versión de \textit{Spark} instalada en el grupo de computadoras fue \textit{Spark} \texttt{2.4.7} y la distribución de \textit{Hadoop} fue \textit{Amazon} \texttt{2.10.1}. Por otro lado, la versión de \textit{Dask} y \textit{Distributed} fue \texttt{2021.4.1}. 


La gestión de recursos y calendarización de tareas se realizó con \textit{YARN} para \textit{Spark} y, para \textit{Dask}, se utilizó el \textit{scheduler} nativo de \textit{Distributed} y el despliegue se realizó de acuerdo a la documentación \cite{daskdistributedsetup}.

El nodo \textit{head} fue una instancia \texttt{m5a.xlarge} de \textit{Amazon EC2} y alojó al proceso \textit{driver} de \textit{Spark} y al proceso \textit{scheduler} de \textit{Dask}. Su trabajo fue administrar la ejecución de los procesos y gestionar los recursos pero no se encargó de ejecutar tareas de procesamiento. Este nodo tuvo las siguientes características de \textit{Hardware}:

\begin{center}
\begin{tabular}{|c|c|}
 \hline
  Procesador & 2.5 GHz AMD EPYC 7000 \\ 
  Núcleos & 4 \\
  Memoria & 16 GB \\ 
  Partición \texttt{root} & 10 GB de \textit{Elastic Block Storage (EBS)}  \\
  Almacenamiento & 64 GB de \textit{EBS}  \\ 
  Sistema Operativo & Amazon Linux 2 \\
  \hline
\end{tabular}
\end{center}

Los nodos \textit{worker} fueron instancias \texttt{m5a.2xlarge} de \textit{Amazon EC2} y se encargaron de ejecutar las tareas asignadas por el \textit{driver} y \textit{scheduler} de \textit{Dask} y \textit{Spark} respectivamente. Los tres nodos \textit{worker} tuvieron las siguientes características de \textit{Hardware}:

\begin{center}
\begin{tabular}{|c|c|}
 \hline
  Procesador & 2.5 GHz AMD EPYC 7000 \\ 
  Núcleos & 8 \\
  Memoria & 32 GB \\ 
  Partición \texttt{root} & 10 GB de \textit{EBS}  \\
  Almacenamiento & 128 GB de \textit{EBS}  \\ 
  Sistema Operativo & Amazon Linux 2 \\
  \hline
\end{tabular}
\end{center}

Se puede encontrar más información sobre las instancias en \cite{ec2-instances}.

La base de datos \textit{MySQL} se creó con el servicio \textit{Relational Database Service} de \textit{AWS} con la versión \texttt{8.0.23} alojada en una instancia \texttt{db.t2.micro} con las siguientes características:

\begin{center}
\begin{tabular}{|c|c|}
 \hline
  Procesador & Intel Xeon \\ 
  Núcleos & 1 \\
  Memoria & 1 GB \\
  Almacenamiento & 20 GB  \\ 
  \hline
\end{tabular}
\end{center}

\section{Datos}

Todas las operaciones se hicieron sobre la base de datos: 
\textit{Reporting Carrier On-Time Performance (1987-present)}, disponible en \cite{linktranstat}. Este conjunto de datos está compuesto de 109 columnas que contienen información de demoras de vuelos dentro de Estados Unidos en el que cada registro corresponde a un vuelo único. Para el experimento se mantuvieron las 109 columnas. Sin embargo, se utilizan únicamente las contenidas en la siguiente lista:

\begin{itemize}
	\item YEAR: (Entero) Año en el que sucedió el vuelo.
	\item QUARTER: (Entero entre 1 y 4) Trimestre en el que sucedió el vuelo.
	\item MONTH: (Entero entre 1 y 12) Mes en el que sucedió el vuelo.
	\item DAY\_OF\_MONTH: (Entero entre 1 y 31) Día en el que sucedió el vuelo.
	\item FL\_DATE: (\textit{string} en formato \textit{yyyy-mm-dd}) Fecha en la que sucedió el vuelo.
	\item TAIL\_NUM: (\textit{string}) La matrícula que identifica de manera única a un avión.
	\item OP\_UNIQUE\_CARRIER: (\textit{string}) Código único de cada aerolínea.
	\item ARR\_TIME: (\textit{string} en formato \textit{hhmm}) Hora de llegada en el horario local.
	\item DEP\_TIME: (\textit{string} en formato \textit{hhmm}) Hora de salida en el horario local.
	\item ARR\_DELAY: (Entero) Diferencia en minutos entre la hora de salida programada y la hora de salida real.
	\item DEP\_DELAY: (Entero) Diferencia en minutos entre la hora de salida programada y la hora de salida real.
	\item ACTUAL\_ELAPSED\_TIME: (\textit{float}) Duración real del vuelo.
	\item TAXI\_IN: (\textit{float}) Tiempo entre el aterrizaje y la llegada del avión a la plataforma de desembarco.
	\item TAXI\_OUT: (\textit{float}) Tiempo en minutos entre que el avión deja la plataforma y el despegue.
	\item ORIGIN: (\textit{string}) Código de 3 letras que identifica a cada aeropuerto de origen.
	\item DEST: (\textit{string}) Código de 3 letras que identifica a cada aeropuerto de destino.
	\item ORIGIN\_CITY\_MARKET\_ID: (Entero) Número que identifica de forma única la ciudad mercado de un aeropuerto de origen. Sirve para unir aeropuertos que sirven al mismo mercado de ciudades.
	\item DEST\_CITY\_MARKET\_ID: (Entero) Número que identifica de forma única la ciudad mercado de un aeropuerto de destino. Sirve para unir aeropuertos que sirven al mismo mercado de ciudades.
\end{itemize}

Los datos utilizados correspondieron al periodo del $1^\text{ro}$ de enero de 2008 al 30 de Noviembre de 2020 y estuvieron compuestos de 80,345,298 registros.

\subsection{Tratamiento de los datos}

Estos datos fueron obtenidos en formato \texttt{csv} y convertidos a \texttt{parquet} para reducir su dimensión y acelerar el tiempo de consulta para ambos \textit{frameworks}. Una vez almacenado en 4096 \textit{parquets}, los datos tuvieron un tamaño de 4.774 GB.

Para evitar que la partición de los datos beneficiara a alguno de las herramientas, se crearon dos copias de los datos. Cada una fue particionada y almacenada de acuerdo a los lineamientos especificados en la documentación de la herramienta correspondiente y con los tipos de datos adecuados para cada \textit{framework}. Los archivos de código utilizados para este propósito están disponibles en \cite{tratamiento-datos}. También se creó el archivo \texttt{\_metadata} que permite acelerar la lectura de los archivos desde \textit{Dask}.

\subsection{Muestras de datos}

Para contrastar las herramientas bajo distintas condiciones, los procesos fueron ejecutados en 5 muestras de datos de los siguientes tamaños:\\

\begin{center}
\begin{tabular}{|ccc|}
  \hline
 Muestra & Número de registros & Tamaño (MB) \\ 
  \hline
  10K & 10,000 & 8 \\ 
  100K & 100,000 & 16 \\ 
  1M & 1,000,000 & 71 \\ 
  10M & 11,008,740 & 2281 \\ 
  Total & 80,345,298 & 4774 \\ 
   \hline
\end{tabular}
\end{center}

La obtención de las muestras se realizó utilizando la función \texttt{sample} de \textit{Spark} sin reemplazo y con la semilla aleatoria $22102001$. Además, para obtener muestras más cercanas al orden de datos deseado, se ordenaron los datos de acuerdo a una columna de enteros aleatorios y después se mantuvo el número deseado con la función \texttt{limit}. En el caso de la muestra \texttt{10M} sólo se usó la función \texttt{sample} con un porcentaje aproximado del total de datos que se buscaba mantener ya que el ordenamiento provocaba errores de memoria tanto en \textit{Dask} como en \textit{Spark}, lo que no permitió recortarlo a exactamente 10,000,000 de registros. Para la muestra \texttt{TOTAL} se mantuvieron todos los registros. El código que se utilizó para este proceso está disponible en \cite{proceso-muestreo}.