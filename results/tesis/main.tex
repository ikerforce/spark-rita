%----------------------------------------------------------------------------------------
%	PREÁMBULO
%----------------------------------------------------------------------------------------

\documentclass[11pt, oneside]{book}
\usepackage[paperwidth=17cm, paperheight=22.5cm, bottom=2.5cm, right=2.5cm]{geometry}

% El borde inferior puede parecerles muy amplio a la vista. Les recomiendo hacer una prueba de impresión antes para ajustarlo

\usepackage{amssymb,amsmath,amsthm} % Símbolos matemáticos
\usepackage[spanish]{babel}
\usepackage[utf8]{inputenc} % Acentos y otros símbolos 
\usepackage{enumerate}
\usepackage{float}% http://ctan.org/pkg/float
\usepackage{hyperref} % Hipervínculos en el índice
\usepackage{graphicx}
\usepackage{amsmath}
\usepackage{algorithm}
\usepackage[noend]{algpseudocode}
%\usepackage{subfig} % Subfiguras
\graphicspath{{Imagenes/}} % En qué carpeta están las imágenes

% Para eliminar guiones y justificar texto
\tolerance=1
\emergencystretch=\maxdimen
\hyphenpenalty=10000
\hbadness=10000

\linespread{1.25} % Asemeja el interlineado 1.5 de Word

\let\oldfootnote\footnote % Deja espacio entre el número del pie de página y el inicio del texto
\renewcommand\footnote[1]{%
\oldfootnote{\hspace{0.05mm}#1}}

\renewcommand{\thefootnote} {\textcolor{Black}{\arabic{footnote}}} % Súperindice a color negro

\setlength{\footnotesep}{0.75\baselineskip} % Espaciado entre notas al pie

\usepackage{fnpos} % Footnotes al final de pág.

\usepackage[justification=centering, font=bf, labelsep=period, skip=5pt]{caption} % Centrar captions de tablas y ponerlas en negritas

\newcommand{\imagesource}[1]{{\footnotesize Fuente: #1}}

\usepackage{tabularx} % Big tables
\usepackage{graphicx}
\usepackage{adjustbox}
\usepackage{longtable}

\usepackage{float} % Float tables

\usepackage[usenames,dvipsnames]{xcolor} % Color

\usepackage{pgfplots} % Gráficas
\pgfplotsset{compat=newest}
\pgfplotsset{width=7.5cm}
\pgfkeys{/pgf/number format/1000 sep={}}

\begin{document}

%----------------------------------------------------------------------------------------
%	PORTADA
%----------------------------------------------------------------------------------------

\title{Confianza institucional, inclusión social, intensidad religiosa y percepción tecnológica como factores de innovación para el crecimiento económico} % Con este nombre se guardará el proyecto en writeLaTex

\begin{titlepage}
\begin{center}

\textsc{\Large Instituto Tecnológico Autónomo de México}\\[2em]

%Figura
\begin{figure}[h]
\begin{center}
\includegraphics[scale=0.50]{itam_logo.png}
\end{center}
\end{figure}

% Pueden modificar el tamaño del logo cambiando la escala

\textbf{\LARGE Comparación de Dask y Spark en el procesamiento de grandes volúmenes de datos}\\[2em]

\textsc{\large Tesis}\\[1em]

\textsc{\large que para obtener el título de}\\[1em]

\textsc{\LARGE Licenciado en Matemáticas Aplicadas}\\[1em]

\textsc{\large Presenta}\\[1em]

\textsc{\LARGE Iker Antonio Olarra Maldonado}\\[1em]

\textsc{\large Asesora}\\[1em]

\textsc{\LARGE Mtra. Liliana Millán}\\[2em]

% Asegúrense de escribir el nombre completo de su asesor

\end{center}

\vspace*{\fill}
\textsc{Ciudad de México \hspace*{\fill} 2021}

\end{titlepage}

%----------------------------------------------------------------------------------------
%	DECLARACIÓN
%----------------------------------------------------------------------------------------

\thispagestyle{empty}

\vspace*{\fill}
\begingroup

\noindent
«Con fundamento en los artículos 21 y 27 de la Ley Federal del Derecho de Autor y como titular de los derechos moral y patrimonial de la obra titulada ``\textbf{Comparación de Dask y Spark en procesamiento de grandes volúmenes de datos}'', otorgo de manera gratuita y permanente al Instituto Tecnológico Autónomo de México y a la Biblioteca Raúl Bailléres Jr., la autorización para que fijen la obra en cualquier medio, incluido el electrónico, y la divulguen entre sus usuarios, profesores, estudiantes o terceras personas, sin que pueda percibir por tal divulgación una contraprestación.»

% Asegúrense de cambiar el título de su tesis en el párrafo anterior

\centering 

\vspace{5em}

\rule[1em]{20em}{0.5pt} % Línea para la fecha

\textsc{Fecha}
 
\vspace{8em}

\rule[1em]{20em}{0.5pt} % Línea para la firma

\textsc{Iker Antonio Olarra Maldonado}

\endgroup
\vspace*{\fill}

%----------------------------------------------------------------------------------------
%	DEDICATORIA
%----------------------------------------------------------------------------------------

\pagestyle{plain}
\frontmatter

\chapter*{}
\begin{flushright}
\textit{A la comunidade de código abierto,\\ por sus aportaciones al progreso de la tecnología.}
\end{flushright}

%----------------------------------------------------------------------------------------
%	AGRADECIMIENTOS
%----------------------------------------------------------------------------------------

\chapter*{Agradecimientos}

\noindent Lorem ipsum dolor sit amet, consectetur adipiscing elit.

% Esta sección es lo único que la gente lee. True story :)

%----------------------------------------------------------------------------------------
%	RESUMEN
%----------------------------------------------------------------------------------------

\chapter*{Resumen}

\noindent El volumen de datos y las capacidades de almacenamiento han aumentado de forma considerable en las últimas décadas. Esto ha incrementado la necesidad de desarrollar \textit{software} de procesamiento de datos que utilice cómputo en paralelo y distribuido para poder procesar los conjuntos masivos de datos creados diariamente. Dos de las herramientas de código abierto más populares para el desarrollo de aplicaciones de ciencia de datos y de ejecución de algoritmos distribuidos son Dask y Spark. El objetivo de esta investigación es comparar su desempeño al realizar operaciones como obtención de estadísticas y ejecución de algoritmos sobre un conjunto de datos grande (43 GB aproximadamente), correspondiente a vuelos dentro de Estados Unidos. Los diferentes procesos se realizarán por ambos \textit{softwares}, en un orden aleatorio y con un número específico de ejecuciones, en un ambiente local y otro en la nube. Además, en ambos algoritmos se implementará una versión paralelizada y distribuida de un algoritmo de ruta mínima. Los resultados serán comparados para determinar bajo qué circunstancias un \textit{software} es superior e identificar las características que permiten ese desempeño.

\pagestyle{plain}

\noindent 

%----------------------------------------------------------------------------------------
%	Summary
%----------------------------------------------------------------------------------------

\chapter*{Summary}

\noindent Data volume and storage capabilities have increased considerably during the past decades. With that in mind, data processing software has leveraged parallelism to keep up with the massive datasets created everyday. Two of the most popular and open-source parallel data processing tools are Dask and Spark. The objective of this research is to compare their performance on a large dataset (8 GB
approximately), corresponding to flights within the United States, while completing different tasks
like computing statistics and performing data processing jobs. The tasks will be executed by both
softwares in an isolated Cloud environment in random order and with a fixed number of executions
for each task. The results will be compared to determine under which circumstances one software is
superior and identify the characteristics that enable such performance.

\pagestyle{plain}

\noindent 

%----------------------------------------------------------------------------------------
%	TABLA DE CONTENIDOS
%---------------------------------------------------------------------------------------

\tableofcontents

%----------------------------------------------------------------------------------------
%	ÍNDICE DE CUADROS Y FIGURAS
%---------------------------------------------------------------------------------------

\listoftables

\listoffigures

%----------------------------------------------------------------------------------------
%	TESIS
%----------------------------------------------------------------------------------------

\mainmatter % Empieza la numeración de las páginas

\pagestyle{plain}

% Incluye los capítulos en el fólder de capítulos

\chapter*{Introducción}
\addcontentsline{toc}{chapter}{Introducción}

% La introducción no cuenta como primer capítulo

\noindent El rápido crecimiento de los datos en los últimos años ha generado la necesidad de crear nuevos sistemas que tengan la capacidad de coleccionar, administrar, procesar y entregar datos que puedan resolver problemas del día a día \cite{seagate}. Los avances en el cómputo han permitido lograrlo, sin embargo, el gran avance en los sistemas de almacenamiento combinados con la desaceleración en el desarrollo de procesadores más rápidos ha obligado a usar el cómputo en paralelo para poder procesar la enorme cantidad de datos creada cada día \cite{sparkguide}. \textit{Dask} y \textit{Spark} son dos herramientas populares que combinan la ejecución en paralelo, cómputo en memoria, evaluación perezosa y calendarizadores dinámicos que les permiten atender problemas de \textit{Big Data} \cite{dask-spark-neuroimaging}.

\textit{Spark} es un motor de cómputo con bibliotecas para procesamiento de datos en paralelo que puede funcionar en un equipo local o escalar a un grupo de computadoras. Soporta múltiples lenguajes de programación como \textit{Python} (aunque está escrito en \textit{Scala}) y es utilizado para aplicaciones como \textit{SQL}, \textit{streaming} y \textit{machine learning} \cite{sparkguide}. Además tiene integración con múltiples sistemas de procesamiento y almacenamiento y tiene desempeño comparable a herramientas diseñadas para propósitos específicos \cite{sparkberkeley}. Esto la convierte en una herramienta versátil y competitiva para el procesamiento de datos. 

\textit{Dask} es una librería desarrollada totalmente en \textit{Python} para cómputo en paralelo que extiende las capacidades de herramientas populares como \textit{NumPy} y \textit{Pandas} a tareas que no caben en memoria o a ambientes distribuidos de cientos de máquinas. Además, cuenta con fuerte integración con proyectos existentes como \textit{Scikit-Learn} que le dan la capacidad de crear aplicaciones de \textit{machine learning} o procesamiento de datos \cite{daskdocs}.

El objetivo de este trabajo es comparar la abstracción más popular (el \textit{DataFrame}) de ambas herramientas bajo diferentes condiciones para determinar bajo qué circunstancias una es superior a la otra. Para esto se ejecutarán diversas tareas que incluyen agregaciones, cálculo de estadísticas, preparación de datos y una implementación del algoritmo \texttt{dijkstra}. Los \textit{frameworks} serán probados en el conjunto de datos \textit{Reporting Carrier On-Time Performance (1987-present)} con un tamaño aproximado de 43GB y las pruebas se ejecutarán en un ambiente local y en uno distribuido en la nube.

El análisis se centrará en evaluar las siguientes 6 capacidades en cada uno: robustez de la herramienta, tiempo de ejecución total, tiempo de cómputo, tiempo de escritura, tiempo de inicio de la ejecución y variación de tiempo registrado. Además, también se incluirá una sección con las dificultades y ventajas encontradas durante la implementación de los procesos en cada herramienta. Esa última sección buscará dar una referencia de bajo qué circunstancias es recomendable adoptar cada una de las herramientas y dar a nuevos usuarios una guía para elegir entre ambas. 

Este trabajo se estructura de la siguiente manera. En el primer capítulo, se hace una revisión de la literatura existente sobre la comparación de ambos \textit{frameworks} y se resumen sus conclusiones más importantes. El segundo capítulo aborda el marco teórico del trabajo, en el que se describe cada uno de los \textit{frameworks} y se explica el algoritmo de ruta mínima \textit{Dijkstra}. En el tercer capítulo, se describe la metodología que se siguió para la ejecución del experimento, describiendo los procesos seleccionados, los ambientes en los que se ejecutó y el conjunto de datos utilizado. El cuarto capítulo tiene como objetivo reportar los resultados de los experimentos y resaltar las conclusiones más importantes derivadas de los mismos. El siguiente capítulo hace un análisis cualitativo sobre la experiencia de implementación utilizando ambos \textit{frameworks}, abordando desde la instalación hasta la configuración de trabajos en cada uno de ellos. Por último, en el capítulo final, se resumen las conclusiones de la investigación y se sugieren temas en los que se podría ahondar como trabajo futuro.

% Se sugiere que el primer párrafo de cada sección no tenga sangría: \noindent


\chapter{Revisión de literatura}

\noindent En esta sección se resumen las investigaciones que se han realizado previamente  

En \cite{comparative-evolution} se hace un análisis de 5 sistemas de \textit{Big Data} entre los cuales se incluyen los \textit{RDDs} de \textit{Spark} y los objetos \textit{futures} y \textit{delayed} de \textit{Dask}. En ambos casos, se trata de abstracciones de más bajo nivel que los \textit{DataFrames} que serán utilizados en esta investigación. El experimiento realizado en \cite{comparative-evolution} tiene un enfoque de procesamiento de imágenes y compara el desempeño de las herramientas en dos procesos y con 6 diferentes tamaños de datos y también en clústeres de 3 tamaños distintos. En el experimento se observa una duración similar entre \textit{Dask} y \textit{Spark} para el primero de los procesos y un desempeño superior de \textit{Dask} en el segundo proceso (en general, \textit{Dask} fue un $14\%$ más rápido que \textit{Spark}), no obstante, \textit{Dask} presenta problemas de falta de memoria para que le impiden completar un proceso con el conjunto de datos de mayor tamaño. Además, con los distintos tamaños de clústeres, \textit{Dask} tiene un desempeño ligeramente mejor que se incrementa junto con el tamaño del clúster. No obstante, para el clúster más pequeño, \textit{Dask} presenta errores de memoria que le impiden completar el proceso. Además, la investigación hace un análisis cualitativo para evaluar la facilidad y efectividad de implementación utilizando cada una de las herramientas y destaca ventajas de \textit{Spark} como el gran tamaño de su comunidad y documentación disponible y la facilidad de implementar funciones de \textit{Python} mediante \textit{User Defined Functions}, además, menciona desventajas como la dificultad de implementación para desarrolladores no familiarizados con la programación funcional, la falta de \textit{caching} de forma predeterminada que puede resultar en recalcular resultados de forma innecesaria y finalmente advierte que afinar los parámetros para el uso de recursos no es trivial. Por otro lado, para \textit{Dask}, destaca como ventajas su fácil instalación y despliegue, su compatibilidad con bibliotecas externas de \textit{Python} y la similitud de su interfaz con otras herramientas populares. Sin embargo, menciona desventajas como la construcción no trivial de la gráfica de ejecución debido a las múltiples formas de hacerlo, la dificultad de encontrar errores en el código (\textit{debugging}) y la inestabilidad que presenta para ejecutar algunos procesos.

Una investigación posterior (\cite{dask-spark-neuroimaging}) hace un comparativo entre \textit{Dask} y \textit{Spark} enfocándose también en procesamiento de imágenes y mediante el uso de abstracciones de más bajo nivel para ambas herramientas (\textit{Futures, Bags} y \textit{Delayed} para \textit{Dask} y \textit{RDDs} para \textit{Spark}). La investigación contempla 3 aplicaciones distintas: la primera compuesta de una operación \textit{map} sencilla, la siguiente compuesta de una operación \textit{map-reduce} y finalmente una operación real también compuesta de una operación \textit{map-reduce}. En el experimento se varía el número de \textit{workers} para ver cómo escalan las aplicaciones, el tamaño de los datos procesados y el número de iteraciones de los procesos. Además, el experimento registra de forma separada el tiempo de lectura, de ejecución, de escritura y de \textit{overhead} de cada aplicación. La conclusión del experimento es que no hay una diferencia sustancial entre ambos \textit{frameworks}.

\newpage



\chapter{Marco teórico}

\noindent El objetivo de este análisis es introducir al cómputo en paralelo y distribuido y describir las capacidades principales de los \textit{frameworks}  de \textit{Big Data}: \textit{Spark} y \textit{Dask}.

\newpage

\section{Introducción al cómputo en paralelo}

\section{Apache Spark}

Apache Spark es un motor de cómputo unificado con bibliotecas para procesamiento de datos en paralelo en clústeres de computadoras. Spark soporta múltiples lenguajes de programación como Python, Java, Scala y R. Entre sus aplicaciones más compunes están trabajos de SQL, \textit{streaming} y \textit{machine learning}.\cite{sparkguide}

\section{Arquitectura de Spark}

Una aplicación de Spark consiste de un proceso \textit{driver} y un conjunto de procesos llamados \textit{executor}. Ambos tipos de procesos trabajan en conjunto durante la ejecución de la aplicación. El proceso \textit{driver} es la parte central de la aplicación, ya que se encarga de mantener información sobre la aplicación, responder a las instrucciones del usuario y analizar, distribuir y calendarizar el trabajo entre los ejecutores. Este proceso existe en uno solo de los nodos y está en constante comunicación con los demás. \cite{sparkguide}. Los \textit{executors}, por otra parte, tienen el objetivo de realizar el trabajo que el \textit{driver} les asigna. Estos se encargan de ejecutan el trabajo y guardan datos para la aplicación \cite{sparkclusteroverview}.

La abstracción de datos principal en Spark son los \textit{Resilient Distributed Datasets (RDDs)} que son colecciones de objetos particionadas en un clúster y que se pueden manipular en paralelo a través de transformaciones. Además, estas transformaciones tienen un \textit{lazy evaluation}, es decir que no realizan la acción inmediatamente, sino que la añaden a un plan eficiente de ejecución. Al llamar una acción, que es una instrucción para calcular un resultado a partir de las transformaciones del plan \cite{sparkguide}.  

\section{Dask}

Dask es una librería de Python para cómputo en paralelo que extiende herramientas populares en el análisis de datos como \textit{NumPy}, \textit{Pandas} y \textit{Python iterators} a tareas que no caben en memoria o a ambientes distribuidos. Además, cuenta con un calendarizador dinámico de tareas y está desarrollado totalmente en \textit{Python}. Se puede utilizar en modo \textit{standalone} o en clústeres con cientos de máquinas. Dask ofrece una interfaz familiar ya que emula a \textit{Numpy} y \textit{Pandas}. 






\chapter{Metodología}

\noindent Este capítulo describe el proceso seguido para comparar el desempeño de \textit{Dask} y \textit{Spark} bajo condiciones homologadas, en dos ambientes, uno local y un cúmulo en la nube. Para tener una mayor visibilidad del desempeño de cada una de ellas, en distintos escenarios, se crearon 8 procesos distintos que consisten en filtración, agrupación, conteos y cálculo de estadísticas sobre los datos seleccionados. Los 8 procesos fueron escritos de forma independiente en \textit{Spark} y \textit{Dask} utilizando funciones equivalentes en ambos \textit{frameworks}. También se incorporó una variación en el tamaño de las muestras para observar el desempeño de cada herramienta con distintos volúmenes de datos. En total se realizaron 14,280 ejecuciones buscando ejecutar 100 veces cada combinación de muestra y ambiente (existieron algunas excepciones que se especifican más adelante debido a falta de recursos o ejecuciones demasiado prolongadas). Cada ejecución fue registrada en una base de datos para su análisis posterior.
\newpage

\section{Procesos para comparación de desempeño}

Para contrastar el desempeño de las herramientas, se implementaron 8 procesos que realizan distintas operaciones sobre el conjunto de datos. Estos procesos buscan someter a los \textit{frameworks} a diferentes escenarios para los que se usan comúnmente estas herramientas. Entre estas operaciones se encuentran las siguientes:
\begin{itemize}
	\item Cálculo de agregaciones sencillas como el conteo de elementos u obtención de los valores máximo y mínimo.
	\item Cálculo de estadísticas como el promedio y la desviación estándar.
	\item Escritura a bases de datos \textit{SQL} o archivos \texttt{parquet}.
	\item Limpieza de datos como eliminación de nulos y borrado de duplicados.
\end{itemize}

Además se implementó un proceso de cálculo de ruta mínima que involucra muchas de las operaciones listadas. En las siguientes secciones se explica de forma más detallada cada uno de los procesos y se listan las operaciones que los componen.

Para ejecutar los datos se creó un \textit{script} que sigue el proceso descrito en el algoritmo \ref{ejecucion_procesos} para la ejecución de los distintos procesos. Este \textit{script} tiene como entradas la muestra de datos que se utilizará (\texttt{muestra}), el número de ejecuciones a realizar (\texttt{n}) y la lista de procesos que se ejecutarán (\texttt{procesos}). Además, dentro del algoritmo, se define la lista \texttt{framework} compuesta de unos y ceros con un orden aleatorio que indica si el proceso será ejecutado con \textit{Dask} (en caso de cero) o con \textit{Spark} (en caso de 1). 

\begin{figure}
\begin{algoritmo}[H]
\caption{Ejecución de procesos}\label{ejecucion_procesos}
\begin{algorithmic}[1]
\Procedure{Ejecuciones(muestra, n, procesos)}{}
\For{\texttt{\textit{proceso} in \textit{procesos}}}
	\State $\textit{framework} \gets \text{lista de 0 y 1 en orden aleatorio de longitud 2\textit{n}}$
	\For{\texttt{\textit{x} in \textit{framework}}}
		\If {\texttt{\textit{x} = 1}}
		\State \texttt{Ejecutar el \textit{proceso} en \textit{Spark}}
		\Else
		\State \texttt{Ejecutar el \textit{proceso} en \textit{Dask}}
		\EndIf
	\EndFor
\EndFor
\EndProcedure
\end{algorithmic}
\end{algoritmo}
\caption{Proceso de ejecución del experimento.}
\end{figure}

De esta forma, cada proceso se ejecuta $n$ veces en cada \textit{framework} y las ejecuciones entre \textit{Spark} y \textit{Dask} se alternan. Además, para iniciar la ejecución de un proceso es necesario que el anterior haya finalizado, lo que evita que compitan por recursos. El algoritmo \ref{ejecucion_procesos} se ejecutó 9 veces, 4 de forma local con tamaños de muestra distintos y 5 en un cúmulo en la nube con los mismos tamaños de muestra que la prueba local y una adicional conformada del total de los datos que no fue ejecutada localmente debido su inestabilidad, que provocaba tiempos de ejecución excesivos o en errores que evitaban la finalización de los procesos. En la mayoría de los casos el valor de $n$ fue $100$, sin embargo, para algunas muestras esta tuvo que ser reducida para procesos específicos debido a que su ejecución era muy tardada o a que esta presentaba errores.

Adicionalmente, para verificar que los resultados de los procesos de \textit{Dask} y \textit{Spark} tuvieran el mismo resultado, se utilizó un \textit{script} que lee el resultado de cada proceso ejecutado en un \textit{framework} y lo compara con su contraparte. De esta forma se asegura que el resultado es el mismo, no obstante, es importante mencionar que para los procesos que utilizan la desviación estándar, sobre una muestra de datos, puede haber diferencias ya que \textit{Dask} y \textit{Spark} generan muestras distintas. El proceso de revisión de resultados se puede consultar en \cite{compara-resultados}.

\subsection{Proceso 1: Cálculo del tamaño de la flota por aerolínea}

Este proceso calcula el número de aviones activos con los que cuenta una aerolínea en un tiempo específico. La agregación se hace por día, mes, trimestre y año. Para obtener el número de equipos únicos se cuenta el número de matrículas distintas correspondientes a cada aerolínea. El proceso se compone de las siguientes operaciones:

\begin{itemize}
	\item Lectura de datos desde \texttt{parquet}.
	\item Borrado de duplicados.
	\item Conteo.
	\item Agregación de datos agrupando por diferentes columnas.
	\item Escritura de datos a \textit{MySQL}.
\end{itemize}

\subsection{Proceso 2: Demoras por aerolínea}

Este proceso tiene como objetivo obtener el promedio de tiempo de demora de cada aerolínea agrupando por las columnas correspondientes a los periodos: año, trimestre, mes y día. Este proceso realiza las siguientes operaciones principales:

\begin{itemize}
	\item Lectura de datos desde \texttt{parquet}.
	\item Conteo.
	\item Cálculo de promedio.
	\item Agregación de datos agrupando por diferentes columnas.
	\item Escritura de datos a \texttt{parquet}.
\end{itemize}

\subsection{Proceso 3: Demora máxima por aeropuerto de origen}

Este proceso calcula el tiempo máximo de demora agrupando por el aeropuerto de origen y obteniendo el resultado para los periodos por año, trimestre, mes y día. El proceso se compone de las siguientes operaciones:

\begin{itemize}
	\item Lectura de datos desde \texttt{parquet}.
	\item Conteo.
	\item Cálculo del máximo.
	\item Agregación de datos agrupando por diferentes columnas.
	\item Escritura de datos a \textit{MySQL}.
\end{itemize}

\subsection{Proceso 4: Demora mínima por aeropuerto de destino}

Este proceso es similar al anterior con la diferencia de que calcula el tiempo mínimo de demora y que hace la agrupación de acuerdo al aeropuerto de destino y obteniendo el resultado para los periodos por año, trimestre, mes y día. El proceso se compone de las siguientes operaciones:

\begin{itemize}
	\item Lectura de datos desde \texttt{parquet}.
	\item Conteo.
	\item Cálculo del mínimo.
	\item Agregación de datos agrupando por diferentes columnas.
	\item Escritura de datos a \texttt{parquet}.
\end{itemize}

\subsection{Proceso 5: Demora promedio por ruta entre aeropuertos}

Este proceso obtiene el tiempo promedio de demora para cada ruta entre los aeropuertos. El primer paso es obtener la ruta que se recorre en cada vuelo y después calcular el promedio del las demoras en el periodo correspondiente.

\begin{itemize}
	\item Lectura de datos desde \texttt{parquet}.
	\item Concatenación de columnas.
	\item Conteo.
	\item Cálculo de promedio.
	\item Agregación de datos agrupando por diferentes columnas.
	\item Escritura de datos a \texttt{parquet}.
\end{itemize}

\subsection{Proceso 6: Desviación estándar de la demora por ruta entre \texttt{market ids}}

Este proceso obtiene la desviación estándar del tiempo de demora para cada ruta entre áreas identificadas por el indicador \texttt{market id} que corresponde a las áreas geográficas cuyos aeropuertos sirven al mismo mercado de personas. El primer paso es obtener la ruta que se recorre en cada vuelo y después calcular la desviación estándar de los indicadores de demoras en el periodo correspondiente.

\begin{itemize}
	\item Lectura de datos desde \texttt{parquet}.
	\item Concatenación de columnas.
	\item Conteo.
	\item Cálculo de la desviación estándar.
	\item Agregación de datos agrupando por diferentes columnas.
	\item Escritura de datos a \textit{MySQL}.
\end{itemize}

\subsection{Proceso 7: Eliminación de nulos}

Este proceso tiene como objetivo eliminar todos los registros que tengan algún valor nulo en cualquiera de las columnas seleccionadas. 

\begin{itemize}
	\item Lectura de datos desde \texttt{parquet}.
	\item Eliminación de nulos.
	\item Conteo.
\end{itemize}

\subsection{Proceso 8: Cálculo de la menor ruta en tiempo utilizando el algoritmo de ruta mínima \textit{Dijkstra}}

El último proceso consiste en calcular la ruta mínima (minimizando el tiempo transcurrido entre la salida del origen y la llegada al destino final) entre dos aeropuertos cualesquiera. Para esto se hizo una implementación del algoritmo \ref{dijkstra_modificado} utilizando \textit{DataFrames} junto con las operaciones predefinidas en cada uno de los \textit{frameworks} (como \texttt{min}, \texttt{orderBy}, \texttt{filter} y operaciones aritméticas) para manipularlos y una función definida por el usuario (\texttt{udf}) que se utilizó para cambiar el formato de la fecha de los vuelos.

El proceso se compone de las siguientes operaciones principales:

\begin{itemize}
	\item Lectura de datos desde \texttt{parquet}.
	\item Filtrado de datos.
	\item Conteo de datos.
	\item Borrado de duplicados.
	\item Cálculo de registro mínimo.
	\item Ordenamiento de datos.
	\item *Almacenamiento temporal de datos mediante \texttt{persist} (\textit{Dask}) y \texttt{checkpoint} (\textit{Spark}).
	\item Suma de columnas a un escalar.
	\item Concatenación de \textit{DataFrames}.
	\item Escritura de datos a \textit{MySQL}.
\end{itemize}

Las operaciones con asterisco varían debido a que se utilizó, para cada \textit{framework}, la operación que mejor desempeño tenía en cuanto a tiempo para almacenar datos intermedios. En el caso de \textit{Spark}, se usó la operación \texttt{checkpoint} ya que se deshace de la parte del plan de ejecución que ya fue ejecutada y almacena los datos intermedios, liberando memoria (a diferencia de \texttt{cache}). Para \textit{Dask}, por otro lado, resultó más conveniente usar la función \texttt{persist} ya que \textit{Dask} no ejecuta otra vez el plan de ejecución completo sino a partir del último \texttt{persist}.

\subsection{Registro de tiempo y resultados}

Al finalizar la ejecución de cada proceso, se obtiene el tiempo de ejecución del mismo (\textit{duration}). Después, esta información se escribe en una base de datos \textit{SQL}¸ junto con los detalles del ambiente en que se ejecutó (\textit{description} y \textit{resources}), la fecha de inicio (\textit{start\_ts}) y término de la ejecución (\textit{end\_ts}) en tiempo UNIX, la fecha de inserción del registro (\textit{insertion\_ts}), el tamaño de muestra en el que se ejecutó el proceso (\textit{sample\_size}), el ambiente en el que se ejecutó el proceso (local o en cúmulo) y el nombre del proceso que se ejecutó (\textit{process}). La siguiente tabla muestra un ejemplo de la información almacenada en la base de datos.

\begin{figure}
\begin{center}
\begin{tabular}{|c|c|}
 \hline
  process & demoras\_ruta\_mktid\_dask \\ 
  start\_ts & 1615231907.4139209 \\
  end\_ts & 1615231916.5688508 \\ 
  duration & 9.154929876327515 \\ 
  description & Ejecucion de prueba en equipo local. \\
  resources & 16 GB y 12 nucleos. \\
  sample\_size & 10K \\
  env & local \\
  insertion\_ts & 2021-03-08 13:31:56 \\
  \hline
\end{tabular}
\end{center}
\caption{Ejemplo de la información que se registra al terminar la ejecución de una prueba del experimento.}
\end{figure}


La información de la ejecución se registra en una de dos tablas dependiendo del \textit{framework} elegido. Los resultados, por otro lado, se almacenan en una tabla específica que se sobreescribe cada que finaliza la ejecución de ese proceso en el caso de \textit{MySQL} o se almacenan en uno o más archivos \texttt{parquet} (dependiendo del número de particiones). 

\section{Infraestructura}

Los \textit{frameworks} fueron comparados en dos ambientes distintos para probar su funcionamiento en modo \textit{standalone} y distribuido en un cúmulo. El primer ambiente es el local y el segundo el cúmulo en la nube. Las siguientes secciones explicarán las características de los ambientes y las ejecuciones en ellas.

\subsection{Ambiente local}

A pesar de que las computadoras personales se quedan cortas para realizar grandes cargas de trabajo de \textit{Big Data}, siguen siendo una herramienta importante para hacer análisis previo y realizar cargas de trabajo no muy grandes. Tanto \textit{Dask} como \textit{Spark} tienen la capacidad de funcionar en una sola máquina y aprovechar los recursos para realizar las cargas de trabajo de forma más rápida que otras herramientas. Por esta razón fue importante probar ambos \textit{frameworks} en un ambiente local. 

Para ello se utilizó un equipo con las siguientes características:

\begin{center}
\begin{tabular}{|c|c|}
 \hline
  Procesador & Intel Core i7-10750H CPU @ 2.60GHz \\ 
  Núcleos & 12 \\
  Memoria & 15.5 GiB \\ 
  Almacenamiento & 128 GB (SSD) + 1 TB (HDD) \\ 
  Sistema Operativo & Ubuntu 20.04.2 LTS \\
  \hline
\end{tabular}
\end{center}

Los procesos fueron ejecutados mediante el proceso descrito en \ref{ejecucion_procesos} y almacenados en una base de datos \textit{MySQL} versión \texttt{8.0.25-0 ubuntu0.20.04.1} alojada en el mismo equipo y compartiendo los recursos listados anteriormente.

Además, se utilizó un ambiente de \textit{Anaconda} con \textit{Python} \texttt{3.8.5}. Este ambiente se utilizó para las ejecuciones con la versión  \texttt{2021.1.1} de \textit{Dask} y la versión \texttt{2.4.7} de \textit{Spark}. 

\subsection{Ambiente en la nube}

El ambiente en la nube tiene como objetivo probar el desempeño de \textit{Dask} y \textit{Spark} en un ambiente distribuido y homogéneo. Para ello se utilizó un cúmulo de \textit{Spark} en \textit{Amazon Elastic Map Reduce (EMR)} de 4 nodos (1 \textit{head} y 3 \textit{worker}). La versión de \textit{Spark} instalada en el cúmulo es \textit{Spark} \texttt{2.4.7} y la distribución de \textit{Hadoop} es \textit{Amazon} \texttt{2.10.1}. Por otro lado, la versión de \textit{Dask} y \textit{Distributed} es \texttt{2021.4.1}. 


La gestión de recursos y calendarización de tareas se realiza con \textit{YARN} para \textit{Spark} y, para \textit{Dask}, se utilizó el \textit{scheduler} nativo de \textit{Distributed} y el despliegue se realizó de acuerdo a la documentación \cite{daskdistributedsetup}.

El nodo \textit{head} es una instancia \texttt{m5a.xlarge} de \textit{Amazon EC2} y aloja al proceso \textit{driver} de \textit{Spark} y el proceso \textit{scheduler} de \textit{Dask}. Su trabajo es administrar la ejecución de los procesos y gestionar los recursos pero no se encarga de ejecutar tareas de procesamiento. Este nodo tiene las siguientes características de \textit{Hardware}:

\begin{center}
\begin{tabular}{|c|c|}
 \hline
  Procesador & 2.5 GHz AMD EPYC 7000 \\ 
  Núcleos & 4 \\
  Memoria & 16 GB \\ 
  Partición \texttt{root} & 10 GB de \textit{Elastic Block Storage (EBS)}  \\
  Almacenamiento & 64 GB de \textit{EBS}  \\ 
  Sistema Operativo & Amazon Linux 2 \\
  \hline
\end{tabular}
\end{center}

Los nodos \textit{worker} son instancias \texttt{m5a.2xlarge}  de \textit{Amazon EC2} y se encargan de ejecutar las tareas asignadas por el \textit{driver} y \textit{scheduler} de \textit{Dask} y \textit{Spark} respectivamente. Los tres nodos \textit{worker} tienen las siguientes características de \textit{Hardware}:

\begin{center}
\begin{tabular}{|c|c|}
 \hline
  Procesador & 2.5 GHz AMD EPYC 7000 \\ 
  Núcleos & 8 \\
  Memoria & 32 GB \\ 
  Partición \texttt{root} & 10 GB de \textit{EBS}  \\
  Almacenamiento & 128 GB de \textit{EBS}  \\ 
  Sistema Operativo & Amazon Linux 2 \\
  \hline
\end{tabular}
\end{center}

Se puede encontrar más información sobre las instancias en \cite{ec2-instances}.

La base de datos \textit{MySQL} se creó con el servicio \textit{Relational Database Service} de \textit{AWS} con la versión \texttt{8.0.23} alojada en una instancia \texttt{db.t2.micro} con las siguientes características:

\begin{center}
\begin{tabular}{|c|c|}
 \hline
  Procesador & Intel Xeon \\ 
  Núcleos & 1 \\
  Memoria & 1 GB \\
  Almacenamiento & 20 GB  \\ 
  \hline
\end{tabular}
\end{center}

\section{Datos}

Todas las operaciones se hicieron sobre la base de datos: 
\textit{Reporting Carrier On-Time Performance (1987-present)}, disponible en \cite{linktranstat}. Este conjunto de datos está compuesto de 109 columnas que contienen información de demoras de vuelos dentro de Estados Unidos en el que cada registro corresponde a un vuelo único. Para el experimento se mantuvieron las 109 columnas, sin embargo se utilizan únicamente las contenidas en la siguiente lista:

\begin{itemize}
	\item YEAR: (Entero) Año en el que sucedió el vuelo.
	\item QUARTER: (Entero entre 1 y 4) Trimestre en el que sucedió el vuelo.
	\item MONTH: (Entero entre 1 y 12) Mes en el que sucedió el vuelo.
	\item DAY\_OF\_MONTH: (Entero entre 1 y 31) Día en el que sucedió el vuelo.
	\item FL\_DATE: (\textit{string} en formato \textit{yyyy-mm-dd}) Fecha en la que sucedió el vuelo.
	\item TAIL\_NUM: (\textit{string}) La matrícula que identifica de manera única a un avión.
	\item OP\_UNIQUE\_CARRIER: (\textit{string}) Código único de cada aerolínea.
	\item ARR\_TIME: (\textit{string} en formato \textit{hhmm}) Hora de llegada en el horario local.
	\item DEP\_TIME: (\textit{string} en formato \textit{hhmm}) Hora de salida en el horario local.
	\item ARR\_DELAY: (Entero) Diferencia en minutos entre la hora de salida programada y la hora de salida real.
	\item DEP\_DELAY: (Entero) Diferencia en minutos entre la hora de salida programada y la hora de salida real.
	\item ACTUAL\_ELAPSED\_TIME: (\textit{float}) Duración real del vuelo.
	\item TAXI\_IN: (\textit{float}) Tiempo entre el aterrizaje y la llegada del avión a la plataforma de desembarco.
	\item TAXI\_OUT: (\textit{float}) Tiempo en minutos entre que el avión deja la plataforma y el despegue.
	\item ORIGIN: (\textit{string}) Código de 3 letras que identifica a cada aeropuerto de origen.
	\item DEST: (\textit{string}) Código de 3 letras que identifica a cada aeropuerto de destino.
	\item ORIGIN\_CITY\_MARKET\_ID: (Entero) Número que identifica de forma única la ciudad mercado de un aeropuerto de origen. Sirve para unir aeropuertos que sirven al mismo mercado de ciudades.
	\item DEST\_CITY\_MARKET\_ID: (Entero) Número que identifica de forma única la ciudad mercado de un aeropuerto de destino. Sirve para unir aeropuertos que sirven al mismo mercado de ciudades.
\end{itemize}

Los datos utilizados corresponden al periodo del $1^\text{ro}$ de enero de 2008 al 30 de Noviembre de 2020 y se componen de 80,345,298 registros.

\subsection{Tratamiento de los datos}

Estos datos fueron obtenidos en formato \texttt{csv} y convertidos a \texttt{parquet} para reducir su dimensión y acelerar el tiempo de consulta para ambos \textit{frameworks} . Una vez almacenado en 4096 \textit{parquets}, los datos tienen un tamaño de 4.774 GB.

Para evitar que la partición de los datos beneficiara a alguno de las herramientas, se crearon dos copias de los datos. Cada una fue particionada y almacenada de acuerdo a los lineamientos especificados en la documentación de la herramienta correspondiente y con los tipos de datos adecuados para cada \textit{framework}. Los archivos de código utilizados para este propósito están disponibles en \cite{tratamiento-datos}. También se creó el archivo \texttt{\_metadata} que permite acelerar la lectura de los archivos desde \textit{Dask}.

\subsection{Muestras de datos}

Para contrastar las herramientas bajo distintas condiciones, los procesos fueron ejecutados en 5 muestras de datos de los siguientes tamaños:\\

\begin{center}
\begin{tabular}{|ccc|}
  \hline
 Muestra & Número de registros & Tamaño (MB) \\ 
  \hline
  10K & 10,000 & 8 \\ 
  100K & 100,000 & 16 \\ 
  1M & 1,000,000 & 71 \\ 
  10M & 11,008,740 & 2281 \\ 
  Total & 80,345,298 & 4774 \\ 
   \hline
\end{tabular}
\end{center}

La obtención de las muestras se realizó utilizando la función \texttt{sample} de \textit{Spark} sin reemplazo y con la semilla aleatoria $22102001$. Además, para obtener muestras más cercanas al orden de datos deseado, se ordenaron los datos de acuerdo a una columna de enteros aleatorios y después se mantuvo el número deseado con la función \texttt{limit}. En el caso de la muestra \texttt{10M} sólo se usó la función \texttt{sample} con un porcentaje aproximado del total de datos que se buscaba mantener ya que el ordenamiento provocaba errores de memoria tanto en \textit{Dask} como en \textit{Spark}, lo que no permitió recortarlo a exactamente 10,000,000 de registros. Para la muestra \texttt{TOTAL} se mantuvieron todos los registros. El código que se utilizó para este proceso está disponible en \cite{proceso-muestreo}.

\chapter{Resultados}

\noindent En este último capítulo se analizan los resultados el experimento en ambos ambientes. 

\newpage

\section{Ambiente local}

\noindent ...

\section{Ambiente en la nube}

\noindent ...

% Gráfica con dos colores a mano


\chapter*{Conclusiones}
\addcontentsline{toc}{chapter}{Conclusiones}

% Las conclusiones tampoco cuentan como capítulo

\noindent Conclusiones sobre la investigación.


%----------------------------------------------------------------------------------------
%	APÉNDICES
%----------------------------------------------------------------------------------------

\begin{appendix}

\chapter{Apéndice}

\section{Formato columnar\label{formatocolumnar}}

\noindent En un formato columnar, cada columna del conjunto de datos se almacena en una localización distinta de disco, normalmente usando bloques de memoria grandes para evitar el escaneo de múltiples bloques. Además, se almacenan en un formato comprimido y se hace explícito cuando se trata de una columna de \textit{IDs} para eficientar la lectura de los datos \cite{column-oriented}. A diferencia del formato por renglón, común en formatos como \texttt{csv}, \texttt{json} o \texttt{avro}, el formato columnar agrupa los datos por columna y no por cada registro. Esto trae las siguientes ventajas: en primer lugar, las consultas sobre una columna específica son más eficientes ya que no es necesario hacer el parseo de cada registro, adicionalmente, se conoce con antelación el tipo de dato que se está procesando dado que normalmente los valores de la columna comparten el tipo de dato, en tercer lugar, se puede lograr una mayor compresión debido a que las columnas normalmente tienen valores similares y además se pueden almacenar de forma separada. Sin embargo, el formato columnar tiene desventajas, sobre todo al hacer búsquedas de un registro específico ya que puede ser mucho más complicado reconstruirlo debido a que las columnas están almacenadas en localizaciones distintas. Los formatos como \texttt{parquet} y \texttt{orc} son los más populares para este tipo de almacenamiento \cite{columnar-storage-blog}.

\section{Caching \label{caching}}

El \textit{caching} consiste en almacenar un subconjunto de datos de forma temporal en una capa de almacenamiento de alta velocidad. De esta forma, cuando estos datos sean requeridos serán obtenidos de forma más rápida que accediendo a la localización original de los datos, o en el caso de motores de procesamiento, volviéndolos a calcular. Así, el uso de datos previamente computados o obtenidos es más eficiente. Normalmente los datos del caché se almacenan en \textit{hardware} de acceso rápido como RAM que tiene capacidades de lectura y escritura más rápidas que otros recursos como aquellos basados en discos duros. El \textit{caching} se utiliza en múltiples aplicaciones que abarcan desde sistemas operativos y bases de datos hasta aplicaciones web e incluye datos obtenidos a partir de consultas a bases de datos, cálculos intensivos computacionalmente, peticiones y respuestas de APIs, entre otros \cite{aws-caching}. 


\end{appendix}

%----------------------------------------------------------------------------------------
%	BIBLIOGRAFÍA
%----------------------------------------------------------------------------------------


\chapter*{Referencias}
\addcontentsline{toc}{chapter}{Referencias}

% Macro. Esto es muy importante, no lo borren

\makeatletter
\renewenvironment{thebibliography}[1]
     {\@mkboth{\MakeUppercase\refname}{\MakeUppercase\refname}%
      \list{}%
           {\setlength{\labelwidth}{0pt}%
            \setlength{\labelsep}{0pt}%
            \setlength{\leftmargin}{\parindent}%
            \setlength{\itemindent}{-\parindent}%
            \@openbib@code
            \usecounter{enumiv}}%
      \sloppy
      \clubpenalty4000
      \@clubpenalty \clubpenalty
      \widowpenalty4000%
      \sfcode`\.\@m}
     {\def\@noitemerr
       {\@latex@warning{Empty `thebibliography' environment}}%
      \endlist}
\makeatother

\begin{thebibliography}{111}

% Lista

% La manera recomendada para citar papers o libros en el formato de Chicago esta en el siguiente vínculo: https://www.chicagomanualofstyle.org/tools_citationguide/citation-guide-2.html

% Es importante poner el apellido del autor seguido del año de publicación, una coma y las páginas consultadas en el texto antes de puntuar y entre paréntesis para las citas en el cuerpo de la tesis

% Ejemplo:

% Las \textit{causas próximas} del crecimiento son conocidas: tecnología, capital humano y físico. La pregunta es ¿por qué unos países sí tienen estas causas próximas y otros no? La respuesta son las \textit{causas fundamentales:} suerte, geografía, cultura e instituciones (Acemoglu 2009, 110).

%AAAAA

\bibitem{linktranstat} On-Time : Reporting Carrier On-Time Performance (1987-present), \url{https://www.transtats.bts.gov/Fields.asp?gnoyr_VQ=FGJ}

\bibitem{daskbestpractices} Dask - Best Practices, \url{https://docs.dask.org/en/latest/dataframe-best-practices.html}

\bibitem{sparkguide} Chambers, B. \& Zaharia, M. (2018). Spark: The definitive guide. Beijing: O'Reilly.

\bibitem{sparkclusteroverview} Cluster mode overview. (n.d.). Retrieved March 24, 2021, from \url{https://spark.apache.org/docs/2.4.0/cluster-overview.html}

\bibitem{sparkberkeley} Zahaira, M., Xin, R. S., Wendell, P., Das, T., Ambrust, M., Dave, A., . . . Stoica, I. (2016, November). Apache Spark: A Unified Engine for Big Data Processing. Retrieved March 24, 2021, from \url{https://people.eecs.berkeley.edu/~alig/papers/spark-cacm.pdf}

\bibitem{daskdocs} Anaconda Inc. (2018). Dask. Retrieved March 26, 2021, from \url{https://docs.dask.org/en/latest/index.html}

\bibitem{daskpaper} Rocklin, M. (2015). Dask: Parallel Computation with Blocked algorithms and Task Scheduling. Retrieved March 26, 2021, from \url{https://pdfs.semanticscholar.org/73b5/8192f30bb6be8e798084d4481b97124570ed.pdf}

\bibitem{daskoverheads} Böhm, S., \& Beránek, J. (2020, October 21). Runtime vs Scheduler: Analyzing Dask’s Overheads. Retrieved March 26, 2021, from \url{https://arxiv.org/pdf/2010.11105.pdf}

\bibitem{daskscheduling} Anaconda Inc. (2018). Scheduling. Retrieved March 26, 2021, from \url{https://docs.dask.org/en/latest/scheduling.html}

\bibitem{daskdistributed} Anaconda Inc. (2016). Dask.distributed. Retrieved March 26, 2021, from \url{https://distributed.dask.org/en/latest/}

\bibitem{sparkibm} Kambhampati, S. (2020, June 30). Explore best practices for Spark performance optimization. Retrieved March 26, 2021, from \url{https://developer.ibm.com/technologies/artificial-intelligence/blogs/spark-performance-optimization-guidelines/}

\bibitem{daskdataframe} Anaconda Inc. (2018). DataFrame. Retrieved April 02, 2021, from \url{https://docs.dask.org/en/latest/dataframe.html}

\bibitem{dijkstraexplicado} Boardman, S. (2014). Shortest path problem (Dijkstra’s algorithm). Retrieved from \url{https://www.pearsonschoolsandfecolleges.co.uk/secondary/Mathematics/16plus/AdvancingMathsForAQA2ndEdition/Samples/SampleMaterial/Chp-02\%20023-043.pdf}

\bibitem{dijkstrabellford} Samah W.G. AbuSalim et. al. (2020). Comparative Analysis between Dijkstra and Bellman-Ford Algorithms in Shortest Path Optimization. Retrieved April 05, 2021, from \url{https://iopscience.iop.org/article/10.1088/1757-899X/917/1/012077}

\bibitem{column-oriented} Abadi, D. J., Boncz, P. A., \&; Harizopoulos, S. (n.d.). Column-oriented Database Systems. Paperhub. \url{https://paperhub.s3.amazonaws.com/741f4f9f3201baa76ddf05d3bff631bd.pdf}

\bibitem{columnar-storage-blog} Rathbone, M. (2019, November 21). Beginners Guide to Columnar File Formats in Spark and Hadoop. Matthew Rathbone's Blog. \url{https://blog.matthewrathbone.com/2019/11/21/guide-to-columnar-file-formats.html}

\bibitem{sparkconfcloudera} Cloudera. (2021, January 21). How-to: Tune Your Apache Spark Jobs (Part 2). Cloudera Blog. https://blog.cloudera.com/how-to-tune-your-apache-spark-jobs-part-2/. 

\bibitem{sparkconfuds} Rao, S. (2020, April 21). Spark performance tuning guidelines. Unified data sciences (Big Data). https://www.unifieddatascience.com/spark-performance-tuning-guidelines. 


%BBBBB

%CCCCC

%DDDDD

%EEEEE

%FFFFF

%GGGGG

%HHHHH

%IIIII

%JJJJJ

%KKKKK

%LLLLL

%MMMMM

%NNNNN

%OOOOO

%PPPPP

%QQQQQ

%RRRRR

%SSSSS

%TTTTT

%UUUUU

%VVVVV

%WWWWW

%XXXXX

%YYYYY

%ZZZZZ

\end{thebibliography}

\newpage
\thispagestyle{empty}
\begin{table}[p]
\centering
\small
\label{ed}
\begin{tabular}{c}
\textit{Comparación de Dask y Spark en} \\ \textit{procesamiento de grandes volúmenes de datos,}\\ escrito por Iker Olarra,\\ se terminó de imprimir en agosto de 2021\\ en los talleres de Tesis Matozo.\\ Campeche 156, colonia Roma,\\ Ciudad de México.
\end{tabular}
\end{table}

% Si lo prefieren, avisen a su taller que esta página ya la incluyeron ustedes para que no les impriman las que ellos usan. Lo recomiendo ampliamente

%%%%%%%%%%%%%%%%%%%%%%%%%%%%%%%%%%%%%%%%%%%%%%%%%%%%%%%%%%%%%%%%%%%%%%%%%%%%%%%%%%%%%%%

\end{document}